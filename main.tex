\documentclass[a4paper,12pt,oneside]{extbook}
\usepackage[english,russian]{babel}
\usepackage{fontspec}
\usepackage{subcaption}
\usepackage{graphicx}
\usepackage{indentfirst}
\usepackage{caption}
\usepackage{wrapfig}
\usepackage{xcolor,soul,lipsum}
\usepackage{amsmath}
\usepackage{amsthm}
\usepackage{hyperref}
\usepackage{enumitem} % no item sep in list
\usepackage[explicit]{titlesec}
\usepackage{amssymb}
\usepackage{titletoc}
\usepackage{tocvsec2}
\usepackage{tocloft}
\usepackage[b]{esvect}
\usepackage{mdframed}
\usepackage{textcomp}
\usepackage{multicol}
\usepackage[%
    left=0.8in,%
    right=0.8in,%
    top=0.8in,%
    bottom=1in,%
]{geometry}%

\DeclareMathOperator{\sign}{sign}

\newmdenv[
    linewidth=2pt,
    align=center,
    topline=false,
    bottomline=false,
    rightline=false,
    skipabove=\topsep,
    skipbelow=\topsep,
]{siderules}


\pagestyle{plain}
\setmainfont{PT Serif}

\renewcommand{\thesection}{\arabic{section}}

\titleformat{\section}
{\Large}{\textbf{\thesection.}}{0.5em}{\textbf{#1}}

\hypersetup{
    colorlinks=true,
    linkcolor=blue,
    filecolor=magenta,
    urlcolor=cyan,
}


\graphicspath{ {./images/} }


\title{
    Конспекты по программированию \\
    \vspace{2cm} 2 семестр \\
    \vspace{2cm} ИКТ \\
    2021 — 2022
    \vfill
}
\author{
    Автор: \\
    Даниил Швалов
}
\date{}

\begin{document}

\begin{titlepage}
    \pagestyle{empty} \cleardoublepage
    \maketitle
    \thispagestyle{empty}
\end{titlepage}

\setcounter{page}{2} { \setcounter{tocdepth}{4} \hypersetup{linkcolor=black}
    \tableofcontents
}

\newpage

\section{Процесс создания ПО}%
\label{sec:Процесс создания ПО}

\textit{Процесс создания ПО} – совокупность мероприятий, целью которых является
создание или модернизация ПО.

\begin{enumerate}
    \item Анализ предметной области (постановка задачи)
    \item Разработка проекта системы
          \begin{enumerate}
              \item Создание модели, отражающей основные функциональные требования,
                    предъявляемые к программе
              \item Выбор метода решения (построение мат. модели)
              \item Разработка алгоритма – последовательности действий по решению
                    задачи
          \end{enumerate}
    \item Реализация программы на языке программирования (кодирование)
    \item Анализ полученных результатов (тестирование)
    \item Внедрение и сопровождение
\end{enumerate}

Этап анализа состоит в исследовании системных требований и проблемы. Различают:
\begin{itemize}
    \item анализ требований — исследование требований к системе;
    \item объектный анализ — исследование объектов предметной области.
\end{itemize}

\section{Жизненный цикл программного продукта}%
\label{sec:Жизненный цикл программного продукта}

\textit{Жизненный цикл} - это непрерывный процесс, который начинается с момента
принятия решения о необходимости его создания и заканчивается в момент его
полного изъятия из эксплуатации. Он базируется на 3-х группах процессов:
\begin{enumerate}
    \item \textbf{Основные процессы} — реализуются под управлением основных сторон
          (заказчик, поставщик, разработчик, оператор и персонал сопровождения),
          вовлеченных в жизненный цикл программных средств. \newline
          \textbf{Примеры}: заказ, поставка, разработка, эксплуатация,
          сопровождение.
    \item \textbf{Вспомогательные процессы} — обеспечивают выполнение основных
          процессов. \newline \textbf{Примеры}: документирование, управление
          конфигурацией, обеспечение качества, верификация, аттестация, совместный
          анализ, аудит, решение проблем.
    \item \textbf{Организационные процессы} — применяются для создания, реализации
          и постоянного совершенствования основной структуры, охватывающей
          взаимосвязанные процессы жизненного цикла и персонал. \newline
          \textbf{Примеры}: управление, создание инфраструктуры,
          усовершенствование, обучение.
\end{enumerate}

\textbf{Стадии жизненного цикла}: замысел \(\longrightarrow\) разработка
\(\longrightarrow\) производство \(\longrightarrow\) применение
\(\longrightarrow\) поддержка \(\longrightarrow\) списание.

\section{Каскадная модель}%
\label{sec:Каскадная модель}

\textbf{Этапы каскадной модели}:
\begin{enumerate}
    \item определение требований;
    \item проектирование;
    \item кодирование, тестирование модулей;
    \item интеграция, тестирования;
    \item эксплуатация, сопровождение.
\end{enumerate}

\textbf{Характеристика модели}:
\begin{itemize}
    \item фиксированный набор стадий;
    \item каждая стадия — законченный результат;
    \item стадия начинается, когда закончилась предыдущая.
\end{itemize}

\textbf{Недостатком} модели является ее «негибкость»:
\begin{itemize}
    \item фаза должна быть закончена, прежде чем приступить к следующей;
    \item набор фаз фиксирован;
    \item тяжело реагировать на изменение требований.
\end{itemize}

Каскадную модель \textbf{рекомендуется использовать} там, где требования заранее
известны и неизменны.

\section{Спиральная модель}%
\label{sec:Спиральная модель}

\textit{Спиральная модель} — частный случай итерационного подхода:
\begin{itemize}
    \item вместо действий с обратной связью — спираль;
    \item отсутствуют заранее фиксированные фазы;
    \item каждый виток спирали — 1 итерация;
    \item каждый виток разбит на 4 сектора:
          \begin{itemize}
              \item определение целей;
              \item оценка и разрешение рисков;
              \item разработка и тестирование;
              \item планирование.
          \end{itemize}
    \item на каждом витке могут применяться разные модели процесса разработки ПО.
\end{itemize}

\textbf{Главное отличие} — акцент на анализ и преодоление рисков.

\section{Agile}%
\label{sec:Agile}

\textit{Agile} — это семейство «гибких» подход к разработке ПО. Agile предполагает, что при реализации проекта
\begin{itemize}
    \item не нужно опираться только на заранее созданные подробные планы;
    \item важно ориентироваться на постоянно меняющиеся условия внешней и внутренней среды;
    \item учитывать обратную связь от заказчиков и пользователей.
\end{itemize}

Частными случаями agile-подходов являются \textit{scrum} и \textit{kanban}.

\section{Kanban}%
\label{sec:Kanban}

\textbf{Особенности}:
\begin{itemize}
    \item kanban — это «подход баланса»;
    \item задача — сбалансировать разных специалистов внутри команды и избежать ситуации, когда дизайнеры работают сутками, а разработчики жалуются на отстутствие новых задач;
    \item вся команда едина — отсутствуют роли владельца продукта и scrum-мастера;
    \item бизнес-процесс делится не на универсальные спринты, а на стадии выполнения конкретных задач (планируется, разрабатывается, тестируется, завершено);
    \item для визуализации используются доски (физ. и эл.) — они позволяют сделать рабочий процесс открытым и понятным для всех специалистов.
\end{itemize}

\textbf{Практики}:
\begin{itemize}
    \item визуализация потока задач;
    \item ограничение невыполненных работ;
    \item управление рабочим потоком;
    \item использование явных правил;
    \item введение петли обратной связи;
    \item улучшение и эволюция.
\end{itemize}

\textbf{Kanban} отлично подходит для небольших проектов, бизнес вебсайтов, где не требуется много времени на планирование. Также он хорошо подходит для долгосрочных проектов, где нет четкой спецификации где задания формируются в процессе разработки.


\section{Scrum}%
\label{sec:Scrum}

\textit{Спринт} — итерация (цикл выпуска продукта):
\begin{itemize}
    \item имеет фиксированную длительность, обычно от 2 до 8 недель;
    \item результат — готовый продукт, который потенциально можно передать заказчику;
    \item в течении спринта делаются все работы по сбору требований, дизайну, кодированию;
    \item рамки спринта фиксированы;
    \item каждый спринт начинается с собрания по его планированию и заканчивается собранием, где подводятся итоги спринта.
\end{itemize}

\textbf{Scrum} подходит для крупного проекта (длительность от 3 месяцев), который имеет полную спецификацию и требования перед началом разработке. В таком случае команда может составить детальный план разработки и весь процесс поделить на спринты.


\end{document}
