\documentclass[a4paper,12pt,oneside]{extbook}
\usepackage[english,russian]{babel}
\usepackage{fontspec}
\usepackage{subcaption}
\usepackage{graphicx}
\usepackage{indentfirst}
\usepackage{caption}
\usepackage{wrapfig}
\usepackage{xcolor,soul,lipsum}
\usepackage{amsmath}
\usepackage{amsthm}
\usepackage{hyperref}
\usepackage{enumitem} % no item sep in list
\usepackage[explicit]{titlesec}
\usepackage{amssymb}
\usepackage{titletoc}
\usepackage{tocvsec2}
\usepackage{tocloft}
\usepackage[b]{esvect}
\usepackage{mdframed}
\usepackage{textcomp}
\usepackage{multicol}
\usepackage[%
    left=0.8in,%
    right=0.8in,%
    top=0.8in,%
    bottom=1in,%
]{geometry}%

\DeclareMathOperator{\sign}{sign}

\newmdenv[
    linewidth=2pt,
    align=center,
    topline=false,
    bottomline=false,
    rightline=false,
    skipabove=\topsep,
    skipbelow=\topsep,
]{siderules}


\pagestyle{plain}
\setmainfont{PT Serif}

\renewcommand{\thesection}{\arabic{section}}

\titleformat{\section}
{\Large}{\textbf{\thesection.}}{0.5em}{\textbf{#1}}

\hypersetup{
    colorlinks=true,
    linkcolor=blue,
    filecolor=magenta,
    urlcolor=cyan,
}


\graphicspath{ {./images/} }


\title{
    Конспекты по программированию \\
    \vspace{2cm} 2 семестр \\
    \vspace{2cm} ИКТ \\
    2021 — 2022
    \vfill
}
\author{
    Автор: \\
    Даниил Швалов
}
\date{}

\begin{document}

\begin{titlepage}
    \pagestyle{empty}
    \cleardoublepage
    \maketitle
    \thispagestyle{empty}
\end{titlepage}

\setcounter{page}{2}
{
    \setcounter{tocdepth}{4}
    \hypersetup{linkcolor=black}
    \tableofcontents
}

\newpage

\section{Процесс создания ПО}%
\label{sec:Процесс создания ПО}

\textit{Процесс создания ПО} – совокупность мероприятий, целью которых является создание или модернизация ПО.

\begin{enumerate}
    \item Анализ предметной области (постановка задачи)
    \item Разработка проекта системы
          \begin{enumerate}
              \item Создание модели, отражающей основные функциональные требования, предъявляемые к программе
              \item Выбор метода решения (построение мат. модели)
              \item Разработка алгоритма – последовательности действий по решению задачи
          \end{enumerate}
    \item Реализация программы на языке программирования (кодирование)
    \item Анализ полученных результатов (тестирование)
    \item Внедрение и сопровождение
\end{enumerate}

Этап анализа состоит в исследовании системных требований и проблемы. Различают:
\begin{itemize}
    \item анализ требований — исследование требований к системе;
    \item объектный анализ — исследование объектов предметной области.
\end{itemize}

\section{Жизненный цикл программного продукта}%
\label{sec:Жизненный цикл программного продукта}

\textit{Жизненный цикл} - это непрерывный процесс, который начинается с момента принятия решения о необходимости его создания и заканчивается в момент его полного изъятия из эксплуатации. Он базируется на 3-х группах процессов:
\begin{enumerate}
    \item
          \textbf{Основные процессы} — реализуются под управлением основных сторон (заказчик, поставщик, разработчик, оператор и персонал сопровождения), вовлеченных в жизненный цикл программных средств.
          \newline
          \textbf{Примеры}: заказ, поставка, разработка, эксплуатация, сопровождение.
    \item
          \textbf{Вспомогательные процессы} — обеспечивают выполнение основных процессов.
          \newline
          \textbf{Примеры}: документирование, управление конфигурацией, обеспечение качества, верификация, аттестация, совместный анализ, аудит, решение проблем.
    \item
          \textbf{Организационные процессы} — применяются для создания, реализации и постоянного совершенствования основной структуры, охватывающей взаимосвязанные процессы жизненного цикла и персонал.
          \newline
          \textbf{Примеры}: управление, создание инфраструктуры, усовершенствование, обучение.
\end{enumerate}

\textbf{Стадии жизненного цикла}: замысел \(\longrightarrow\) разработка \(\longrightarrow\) производство \(\longrightarrow\) применение \(\longrightarrow\) поддержка \(\longrightarrow\) списание.

\section{Каскадная модель}%
\label{sec:Каскадная модель}

\textbf{Этапы каскадной модели}:
\begin{enumerate}
    \item определение требований;
    \item проектирование;
    \item кодирование, тестирование модулей;
    \item интеграция, тестирования;
    \item эксплуатация, сопровождение.
\end{enumerate}

\textbf{Характеристика модели}:
\begin{itemize}
    \item фиксированный набор стадий;
    \item каждая стадия — законченный результат;
    \item стадия начинается, когда закончилась предыдущая.
\end{itemize}

\textbf{Недостатком} модели является ее «негибкость»:
\begin{itemize}
    \item фаза должна быть закончена, прежде чем приступить к следующей;
    \item набор фаз фиксирован;
    \item тяжело реагировать на изменение требований.
\end{itemize}

Каскадную модель \textbf{рекомендуется использовать} там, где требования заранее известны и неизменны.


\end{document}
